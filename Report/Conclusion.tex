\fancyhead[R]{Leonie Boland}

\subsection{Deep Q-Learning Agent}

In summary, the approach of using Deep Q-learning to play Bomberman with reinforcement learning was successful. Once the main framework was in place, we found that it is very important to give a lot of thought into the rewarding and auxiliary rewards as the performance of the agent depends crucially upon that. Certainly, fine tuning hyperparameters also plays an important role, but as soon as the model was running well, there seemed to be no big differences in the performance based on hyperparameter changes.

The model does not provide a perfect performance as explored in the end of section \ref{sec:deepqexperiments}. With the given time constraints we are only able to give hypotheses for the reasons of the malfunctions. Our believe is that, most of the deficits might resolve when working out a better way to implement the state to feature function or the auxiliary rewards.

\subsubsection{Outlook}

The Deep Q-learning method has proved itself to be a powerful reinforcement learning technique to create a well playing bomberman agent. Through a series of experiments and analyses specified in \ref{sec:deepqexperiments} we took actions in respect of improving on the basic framework of our model. Despite the promising results, there are a few challenges that remained and could be embarked upon when investing more time. \\

The most obvious flaw of our agent is that it can get stuck in a loop, for example just moving up and down, even though there are many better moves to do. We doubt that this could simply be corrected by more training as we also observed that the rule-based agent with a restricted vision has similar problems. There are at least three ways to improve the model in regards to this challenge. First, we could give the agent a larger vision field, which would probably lead to a longer computation time, as the neural network would get more information as input. On the other hand more work could be done on the outside map to give it more relevance in the case that there are no important items left in the agents vision field. And lastly, the agent could start moving in a random way without dropping bombs, to explore the rest of the field. As soon as the agent "sees" a coin, crate or opponent it should act appropriate to that scenario again.

Furthermore the computation of the features given to the neural network is rather minimalist. We could take advantage of what we know from the game state. For example if the agent is in the explosion radius of a bomb we could feed the network with that information, so that it can learn more easily that it needs to move away. However, if going too far with this approach the agent more or less could end up as a rule-based agent rather than a reinforcement learning agent. This is a fine line but our implementation could be improved in that sense without making rule-based decisions.\\ 

Although the auxiliary rewards have been a very present topic during the development already they still need to be mentioned as a possibility to further enhance the DQN agent. To perfect the rewards one would need to act out even more scenarios in theory and think about how to reward the events in comparison to each other so that the agent can rely on these rewards. Especially in regards to the movement of the agent, which is only rewarded in terms of the distance to the coin, more fine tuning is needed. In the classic environment with a lot of crates and only a few coins it is not unlikely that there is no coin on the field at all. Therefore, the agent should be triggered to move in a promising direction not only because of a coin but also due to crates or opponents. This is only one concrete example but spending even more time to figure out the auxiliary rewards would probably be advisable to reach a strong Bomberman agent model. \\

We went with a very basic neural network architecture only including one convolution and one hidden linear layer for each feature map before concatenating them into one network, which is followed by one hidden linear layer again. As there is no guideline how to create the most efficient neural network for one specific task it is a common practice to test different neural network structures and parameters in the different layers. We chose to have parallel computed convolution layers for the different feature maps but we could have taken all 7 $\times$ 7 feature maps as input to one convolution layer with more channels, in our case 3 channels. This probably would give a similar result to our original computation but is considered to be more efficient computing. However, we neglected to experiment with the number of channels or with adding more convolution layers stringed together. Without testing different features in the network's architecture, like the number of layers or channels, we cannot be sure that we have found the best neural network to solve the problem at hand.

Experimenting with more sophisticated neural network features, for example attention mechanisms, can help to capture long-term dependencies within the Bomberman game. Currently attention mechanisms are most popular in language model, but it could underline the importance of certain actions in a specific environment while suppressing irrelevant information. This might be hard to integrate into our current model, thus probably asks for a new approach, at least in regards to the neural network.
