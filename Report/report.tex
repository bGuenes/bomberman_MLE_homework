% Festlegung des Allgemeinen Dokumentenformats
\documentclass[a4paper,12pt,headsepline, fleqn, english]{scrartcl}%article

% Umlaute unter UTF8 nutzen
\usepackage[utf8]{inputenc}

% Variablen
%\input{latex_einstellungen/variablen}

% weitere Pakete
% Grafiken aus PNG Dateien einbinden
\usepackage{graphicx}

% Deutsche Sonderzeichen und Silbentrennung nutzen
%\usepackage[ngerman]{babel}

% Eurozeichen einbinden
\usepackage[right]{eurosym}

% Zeichenencoding
\usepackage[T1]{fontenc}

\usepackage{lmodern}

\usepackage{algpseudocode}
\usepackage{algorithm}
\let\oldReturn\Return
\renewcommand{\Return}{\State\oldReturn}

\usepackage{float}
% floatende Bilder ermöglichen
%\usepackage{floatflt}
\usepackage[font=footnotesize]{caption}
%\usepackage{caption}
\usepackage{subcaption}

% mehrseitige Tabellen ermöglichen
\usepackage{longtable}

% Unterstützung für Schriftarten
%\newcommand{\changefont}[3]{ 
	%\fontfamily{#1} \fontseries{#2} \fontshape{#3} \selectfont}

% Packet für Seitenrandabständex und Einstellung für Seitenränder
\usepackage{geometry}
\geometry{left=3.5cm, right=2cm, top=2.5cm, bottom=2.5cm}

% Paket für Boxen im Text
\usepackage{fancybox}

% bricht lange URLs "schön" um
\usepackage[hyphens,obeyspaces,spaces]{url}

% Paket für Textfarben
\usepackage{color}

\usepackage{hyperref}
\hypersetup{
	colorlinks,
	citecolor=black,
	filecolor=black,
	linkcolor=black,
	urlcolor=black
}

% Mathematische Symbole importieren
\usepackage{amssymb}
\usepackage{amsmath}
% auf jeder Seite eine Überschrift (alt, zentriert)
%\pagestyle{headings}

% erzeugt Inhaltsverzeichnis mit Querverweisen zu den Abschnitten (PDF Version)
%\usepackage[bookmarksnumbered,pdftitle={\titleDocument},hyperfootnotes=false]{hyperref}
%\hypersetup{colorlinks, citecolor=red, linkcolor=blue, urlcolor=black}
%\hypersetup{colorlinks, citecolor=black, linkcolor= black, urlcolor=black}

% neue Kopfzeilen mit fancypaket
\usepackage{fancyhdr} %Paket laden
\pagestyle{fancy} %eigener Seitenstil
\fancyhf{} %alle Kopf- und Fußzeilenfelder bereinigen
\fancyhead[L]{\nouppercase{\leftmark}} %Kopfzeile links
\fancyhead[C]{} %zentrierte Kopfzeile
\fancyhead[R]{Leonie Boland} %Kopfzeile rechts
\renewcommand{\headrulewidth}{0.4pt} %obere Trennlinie
\fancyfoot[C]{\thepage} %Seitennummer
%\renewcommand{\footrulewidth}{0.4pt} %untere Trennlinie

%authoryear-comp
\RequirePackage[style = numeric, backend = biber]{biblatex}
\addbibresource{SeminarMML.bib}

% Paket für Zeilenabstand
\usepackage{setspace}



\begin{document}
	\thispagestyle{empty}

\begin{verbatim}
	
	
	
\end{verbatim}

\begin{center}
	\Large{\textbf{Heidelberg University}}
\end{center}


\begin{center}
	\Large{\textbf{End of the Semester Project}}\\
	\Large{Lecture: Machine Learning Essentials}
\end{center}

\begin{verbatim}
	
	
	
\end{verbatim}

\begin{center}
	\large{\textbf{Report}}\\
	\huge{Using Reinforcement Learning Methods to Train an Agent to Play Bomberman}	
\end{center}

\begin{verbatim}
	
	
	
	
	
	
	
	
	
	
	
	
	
	
	
	
\end{verbatim}

% Links unten die wichtigen Daten
\begin{flushleft}
	\begin{tabular}{llll}
		\textbf{Institute:} && Computer Vision and Learning Lab & \\
		&& Biomedical Genomics Group& \\
		&& \\
		\textbf{Authors:} & & Leonie Boland, MatNr. 4055040& \\
		& & Kevin Klein, MatNr. 4114347& \\
		& & Berkay Günes, MatNr. 3446492& \\
		& &\\
		\textbf{Version from:} & & \today &\\
		& & \\
		\textbf{Supervisor:} & & Prof. Dr. Ullrich Köthe &\\
	\end{tabular}
\end{flushleft}
	
	% römische Numerierung
	\pagenumbering{roman}
	
	% Neue Seite für die Erklärung
	\newpage
	
	\section{Declaration}
	
	\newpage
	
	% Seitenzählung bei Inhaltsverzeichnis beginnen
	\setcounter{page}{1}
	
	% Inhaltsverzeichnis anzeigen
	\thispagestyle{empty}
	\tableofcontents
	
	\newpage
	
	% arabische Seitennummerierung ab hier
	\pagenumbering{arabic}
	
	\section{Introduction}
	
	\section{Methods}
	
	\section{Training}
	
	\section{Experiments and Results}
	
	\section{Conclusion}
	
	\newpage	
	
	%\nocite{*}
	\printbibliography
\end{document}