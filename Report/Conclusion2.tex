\fancyhead[R]{Kevin Klein}
\subsection{Our second best agent}

In conclusion, our attempt to develop an agent that initially imitated the rule-based agent and then enhance its capabilities using Q-learning led 
to mixed results. The training to imitate the rule-based agent was more or less successful, and the agent demonstrated promising performance, aligning with the rule-based 
strategies in a certain point. However, when we introduced Q-learning to further improve its gameplay, the results were disappointing. Instead of enhancing its skills, 
the agent's performance steadily declined during training, reaching a point where it was unable to play the game proficiently.

We hypothesized that a significant factor contributing to this issue was the disparity in the dimensionality of the training sets 
used for imitation and Q-learning. The transition from the well-structured imitation training data to the Q-learning dataset, which likely 
had different dimensions and characteristics, might have caused confusion and hindered the agent's learning process. This mismatch in the training sets 
could have led to the deterioration in performance, highlighting the importance of consistent and compatible data when implementing reinforcement learning techniques. 
Further investigations and refinements in the training process are necessary to address these challenges and achieve successful integration of imitation learning and
Q-learning for optimal agent performance in the game.